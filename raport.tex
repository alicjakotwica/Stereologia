\documentclass[a4paper,12pt]{article}

\input{documentTemplate.tex}
\newcommand{\labdate}{\today}
\newcommand{\temat}{Echo}

\begin{document}
    
\input{tabela.tex}


\section{Cel ćwiczenia}

\section{Wprowadzenie}

\section{Aparatura i materiały}

\section{Wyniki i obliczenia}

\subsection{Udział objętościowy}

$$V_V=\cfrac{\sum _i P_{ik}}{P_k}$$

Gdzie:

$V_V$ - udział objętościowy,

$P_{ik}$ - suma punktów w fazie,

$P_k$ - suma wszystkich punktów.

\subsection{Rozwinięcie powierzchni $S_V$}

$$S_V=\cfrac{\sum ^n_{i=1} S_i}{V}$$

Gdzie:

$S_V$ - rozwinięcie powierzchni,

$S_i$ - suma powierzchni,

$V$ - objętość, jaką zajmują ziarna.

$$\overline{P}_L=\cfrac{p}{k\cdot l}\sum^k_{i=1}n_i$$

Gdzie:

$\overline{P}_L$ - średnia liczba przecięć na jednostkę długości 

$p$ - powiększenie

$k$ - liczba pomiarów

$l$ - długość

$n_i$ - liczba punktów przecięcia z granicą faz

\subsection{Wielkość ziaren}

$$N_A=\cfrac{p^2}{a\cdot b}(z+0.5w+0.25u)$$

Gdzie:

$N_A$ - liczba ziaren mierzonych na jednostkowej powierzchni zgładu,

$p$ - powiększenie mikroskopu,

$a$ i $b$ - długość boków prostokąta,

$z$ - liczba ziaren leżących całkowicie wewnątrz prostokąta,

$w$ - liczba ziaren przeciętych przez brzegi prostokąta,

$u$ - liczba ziaren leżących w narożach prostokąta.


\section{Podsumowanie i wnioski}

\section{Załączniki wyników}

\end{document}
