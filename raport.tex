\documentclass[a4paper,12pt]{article}

\input{documentTemplate.tex}
\newcommand{\labdate}{\today}
\newcommand{\temat}{Echo}

\begin{document}
    
\input{tabela.tex}


\section{Cel ćwiczenia}

Wyznaczenie udziału objętościowego, rozwinięcie powierzchni oraz wielkości ziaren.

\section{Wprowadzenie}

Stereologia to nauka o przestrzennej budowie materiałów i metodach ich ilościowego opisu w oparciu o pomiary wykonane na płaskich przekrojach materiałów. Podstawowe parametry charakteryzujące elementy mikrostrukturę tworzywa to udział objętościowy $V_V$, rozwinięcie powierzchni $S_V$ oraz wielkość ziaren $N_A$.

Wyróżnia się 2 metody wyznaczania udziału objętościowego faz $V_v$, które wynikają z równania stereologicznego:

$$V_V=A_A=L_L=P_P$$

\begin{enumerate}
    \item Metoda liniową, która polega na pomiarze sumy długości  odcinków przypadających na daną fazę i odniesieniu jej do całkowitej długości linii przyłożonej do analizowanego zgładu. Metoda pracochłonna, ale przydatna przy określaniu udziału objętościowego faz znajdujących się w małych ilościach i liniowo zorientowanych. 
    \item Metoda punktową, która opiera się na rachunku prawdopodobieństwa  prawdopodobieństwo trafienia punktu rzuconego losowo na powierzchnię zgładu w daną fazę jest równa stosunkowi powierzchni zajmowanej przez tą fazę do całej powierzchni zgładu i nie zależy od kształtu oraz rozmieszczenia badanej fazy. Wyróżnia się 2 warianty tej metody – metodę punktów losowych oraz metodę punktów rozłożonych systematycznie (metoda siatkowa).
\end{enumerate}

\newpage

Do wyznaczenia rozwinięcia powierzchniowego wyróżnia się 2 metody :

\begin{enumerate}
    \item  Metoda siecznych przypadkowych, Sieczną o długości ”l” rzuca się przypadkowo ”k” razy na fotografię o powiększeniu ”p” i zlicza się liczbę przecięć siecznej ”n” z granicami ziaren, z granicami ziarno - por w każdym rzucie. Średnia liczba przecięć na jednostkę długości $P_L$.
    \item  Metoda siecznych skierowanych,odpowiednio do kierunków zorientowanych osi czy płaszczyzn badanego materiału, można badać zarówno tekstury izomeryczne jak również zorientowane. W tym przypadku można wyznaczyć średnią liczbę przecięć granic ziaren na jednostkę długości siecznych przy niewielkiej liczbie zgładów, a ze znajomości $P_L$ wyznaczymy rozwinięcie powierzchni jakie przypada na jednostkę objętości. Najbardziej poglądową charakterystykę orientacji układu linii na płaszczyźnie, a zarazem bardzo czułym wskaźnikiem istnienia nawet niewielkiego zorientowania linii jest tzw. „róża liczby przecięć” zaproponowana przez Sołtykowa. Przedstawia ona zależność między średnią liczbą przecięć na jednostkową długość siecznych a kierunkiem siecznych we współrzędnych biegunowych.
\end{enumerate}

W celu wyznaczenia wielkości ziaren  stosujemy :

\begin{enumerate}
    \item Metoda odwrotności średnic Sołtykowa, która polega na zmierzeniu średnicy wszystkich przekrojów badanych ziaren znajdujących się na powierzchni zgładu.Pomiary średnicy dla każdego ziarna wykonać należy  w dwu wzajemnie prostopadłych kierunkach.
    \item Metoda Jeffriesa, która jest najprostsza. Na obraz zgładu nanosimy prostokąt o bokach a i b, który dzieli nam ziarna na trzy grupy: z, w i u.
\end{enumerate}


Literatura

Metody Badań „Mikroskopia Optyczna” część B, Jan Piekarczyk

\newpage

\section{Aparatura i materiały}

Aparatura:
\begin{itemize}
    \item komputer
\end{itemize}

Materiały:
\begin{itemize}
    \item linijka,
    \item kątomierz,
    \item kalka techniczna,
    \item mikrostruktura kompozytu włókno C/żywica epoksydowa
    \item mikrostruktura grafitu $p=300x$
\end{itemize}


\section{Wyniki i obliczenia}

\subsection{Udział objętościowy $V_V$}

Na kalce narysowałyśmy siatkę o oczkach $1.5$ x $1.5 cm$. Następnie przyłożyłyśmy siatkę do wydrukowanej kartki i zaznaczyłyśmy przecięcia z faza włókna C/żywica epoksydowa, na tej podstawie wyznaczyłyśmy udziały objętościowe faz i minimalną ilość przyłożeń siatki. Pomiary powtórzyłyśmy jeszcze 9 razy używając różnych wartości węzłów. Podobne czynności przeprowadzono dla metody liniowej. Odcinek o długości $10cm$ przykładałyśmy w przypadkowych miejscach, a następnie zliczyłyśmy punkty przecięcia. 

Na podstawie otrzymanych wyników obliczyłyśmy udział objętościowy fazy $\alpha$, a następnie obliczyłyśmy szacunkową minimalną wartość przyłożeń siatki ze wzoru:

$$n\ge \cfrac{t^2 \cdot (1-V_V(\alpha))}{\gamma^2 \cdot k \cdot V_V(\alpha)}$$

gdzie:

$n$ - szacunkowa minimalna liczba przyłożeń siatki

$k$ - liczba węzłów siatki

$V_V(\alpha)$ - udział objętościowy fazy $\alpha$

$\gamma$ - błąd względny wynoszący $0.05$, dla prawdopodobieństwa $95\%$ $(t=1.96)$
\newpage

$$\mu=\overline{V}\pm t \cdot s(\overline{V})=\overline{V}\pm \cfrac{t\cdot s}{\sqrt{n}}=\overline{V}\pm \delta = \overline{V}\pm \gamma \cdot \overline{V}$$

Gdzie:

$\gamma=\cfrac{t\cdot s}{\overline{V}\cdot \sqrt{n}}$

$\overline{V}$ - wyliczona średnia dla próbki pobranej populacji

$s(\overline{V})$ - oszacowany błąd standardowy

$s$ - oszacowane odchylenie standardowe

$x$ - wartości mierzone 

$n$ - liczba pomiarów

$t$ - zmienna standaryzowana ($t=1.96$ dla prawdopodobieństwa $95\%$ $\alpha=0.05$)

$\delta$ - błąd bezwzględny

$\gamma$ - błąd względny

$s=\sqrt{\cfrac{\Sigma (V-\overline{V})^2}{n-1}}$

\subsubsection{Metoda punktowa}

Po dokonanych obliczeniach wiemy, że szacunkowa minimalna wartość przyłożeń siatki wynosi $n\ge 193$.

% Please add the following required packages to your document preamble:
% \usepackage{graphicx}
\begin{table}[H]
\centering
\caption{Udział objętościowy fazy $\alpha$ dla metody punktowej.}
\label{tab:my-table}
\resizebox{0.7\textwidth}{!}{%
\begin{tabular}{|r|r|r|r|r}
\hline
\multicolumn{1}{|c|}{L.P} & \multicolumn{1}{c|}{$z(\alpha)$} & \multicolumn{1}{c|}{$k$} & \multicolumn{1}{c|}{$V_V(\alpha)$} & \multicolumn{1}{c|}{$(V-\overline{V}$)} \\ \hline
1                         & 6                                & 42                       & 0,14                               & \multicolumn{1}{r|}{-0,02}                               \\ \hline
2                         & 8                                & 42                       & 0,19                               & \multicolumn{1}{r|}{0,03}                                \\ \hline
3                         & 6                                & 42                       & 0,14                               & \multicolumn{1}{r|}{-0,02}                               \\ \hline
4                         & 7                                & 42                       & 0,17                               & \multicolumn{1}{r|}{0,01}                                \\ \hline
5                         & 7                                & 42                       & 0,17                               & \multicolumn{1}{r|}{0,01}                                \\ \hline
6                         & 8                                & 42                       & 0,19                               & \multicolumn{1}{r|}{0,03}                                \\ \hline
7                         & 5                                & 42                       & 0,12                               & \multicolumn{1}{r|}{-0,04}                               \\ \hline
8                         & 5                                & 42                       & 0,12                               & \multicolumn{1}{r|}{-0,04}                               \\ \hline
9                         & 7                                & 42                       & 0,17                               & \multicolumn{1}{r|}{0,01}                                \\ \hline
10                        & 8                                & 42                       & 0,19                               & \multicolumn{1}{r|}{0,03}                                \\ \hline
\multicolumn{3}{|c|}{Wartość średnia $\overline{V}$}                   & 0,16                               & \multicolumn{1}{l}{}                                     \\ \cline{1-4}
\multicolumn{3}{|c|}{Odchylenie standardowe $s$}                                        & 0,03                               & \multicolumn{1}{l}{}                                     \\ \cline{1-4}
\end{tabular}%
}
\end{table}

Przedział ufności dla metody punktowej wynosi $\mu=0.16\pm 0.0541$.

Błąd względny wynosi $\gamma=0.1073$.

\subsubsection{Metoda liniowa}

Po dokonanych obliczeniach wiemy, że szacunkowa minimalna wartość przyłożeń linii wynosi $n\ge 51$.

% Please add the following required packages to your document preamble:
% \usepackage{graphicx}
\begin{table}[H]
\centering
\caption{Udział objętościowy fazy $\alpha$ dla metody liniowej}
\label{tab:my-table}
\resizebox{0.7\textwidth}{!}{%
\begin{tabular}{|r|r|r|r|r}
\hline
\multicolumn{1}{|l|}{L.P} & \multicolumn{1}{l|}{$z(\alpha)$ {[}mm{]}} & \multicolumn{1}{l|}{$k$ {[}mm{]}} & \multicolumn{1}{l|}{$V_V(\alpha)$} & \multicolumn{1}{l|}{$(V-\overline{V})$} \\ \hline
1                         & 15                                        & 100                               & 0,15                               & \multicolumn{1}{r|}{-0,08}                               \\ \hline
2                         & 25                                        & 100                               & 0,25                               & \multicolumn{1}{r|}{0,02}                                \\ \hline
3                         & 27                                        & 100                               & 0,27                               & \multicolumn{1}{r|}{0,04}                                \\ \hline
4                         & 29                                        & 100                               & 0,29                               & \multicolumn{1}{r|}{0,06}                                \\ \hline
5                         & 24                                        & 100                               & 0,24                               & \multicolumn{1}{r|}{0,01}                                \\ \hline
6                         & 12                                        & 100                               & 0,12                               & \multicolumn{1}{r|}{-0,11}                               \\ \hline
7                         & 28                                        & 100                               & 0,28                               & \multicolumn{1}{r|}{0,05}                                \\ \hline
8                         & 27                                        & 100                               & 0,27                               & \multicolumn{1}{r|}{0,04}                                \\ \hline
9                         & 28                                        & 100                               & 0,28                               & \multicolumn{1}{r|}{0,05}                                \\ \hline
10                        & 17                                        & 100                               & 0,17                               & \multicolumn{1}{r|}{-0,06}                               \\ \hline
\multicolumn{3}{|l|}{Wartość średnia $\overline{V}$}                                     & 0,23                               & \multicolumn{1}{l}{}                                     \\ \cline{1-4}
\multicolumn{3}{|l|}{Odchylenie standardowe $S$}                                                          & 0,06                               & \multicolumn{1}{l}{}                                     \\ \cline{1-4}
\end{tabular}%
}
\end{table}

Przedział ufności dla metody punktowej wynosi $\mu=0.23\pm 0.1211$.

Błąd względny wynosi $\gamma=0.1651$.


\subsection{Rozwinięcie powierzchni $S_V$}

Na kartce narysowaliśmy linię $10cm$, a następnie przyłożyłyśmy do zdjęcia i policzyłyśmy ilość przeciętych granic międzyziarnowych. Później wykonałyśmy poniższe obliczenia, a wyniki zestawiłyśmy w tabeli 4.  Natomiast w metodzie siecznych skierowanych na kartce narysowaliśmy linie o długości $5cm$. Następnie za pomocą kątomierza narysowano 5 linii, każda o długości $5cm$, pod kątami $30^\circ$, $60^\circ$, $90^\circ$, $120^\circ$ i $150^\circ$. Przyłożono do fotografii i zmierzono 10-krotnie liczbę przecięć z każdą linia. Wyniki zanotowano w tabeli 2.
Powiększenie zdjęcia  wynosi 300x , które zastosowano w obliczeniach.
\newline

Przedział ufności dla obydwóch pomiarów został policzony ze wzoru:
$$\mu = S_V\pm t\cdot s$$
\newline 

Błąd względny dla obydwóch pomiarów policzono ze wzoru:

$$\gamma=\cfrac{t\cdot s}{S_V\cdot \sqrt{n}}$$

\subsubsection{Pomiar rozwinięcia powierzchni}

W celu obliczenia ilości przecięć na jednostkę długości odcinka wykorzystaliśmy wzór:

$$P_L=\cfrac{P}{L}$$

Gdzie:

$P$ - ilość przeciętych granic międzyziarnowych

$L$ - długość odcinka w cm.
\newline

Do policzenia rozwinięcia powierzchni wykorzystaliśmy wzór:

$$S_V=\cfrac{2p}{k\cdot l} \sum ^k_{i=1}n_i$$

gdzie:

$p$ - powiększenie

$k$ - liczba pomiarów

$l$ - długość odcinka

$n_i$ - liczba przecięć z granicą faz

% Please add the following required packages to your document preamble:
% \usepackage{graphicx}
\begin{table}[H]
\centering
\caption{Rozwinięcie powierzchni.}
\label{tab:my-table}
\resizebox{0.7\textwidth}{!}{%
\begin{tabular}{|c|l|r|r|r|}
\hline
L.P & \multicolumn{1}{c|}{$P$} & \multicolumn{1}{c|}{$L$ {[}cm{]}} & \multicolumn{1}{c|}{$P_L$ $[cm^{-1}]$} & \multicolumn{1}{c|}{$S_V$ $[cm^2/cm^3]$} \\ \hline
1   & 17                       & 10                                & 1,7                                    & 1020                                   \\ \hline
2   & 17                       & 10                                & 1,7                                    & 1020                                   \\ \hline
3   & 26                       & 10                                & 2,6                                    & 1560                                   \\ \hline
4   & 22                       & 10                                & 2,2                                    & 1320                                   \\ \hline
5   & 12                       & 10                                & 1,2                                    & 720                                    \\ \hline
6   & 16                       & 10                                & 1,6                                    & 960                                    \\ \hline
7   & 9                        & 10                                & 0,9                                    & 540                                    \\ \hline
8   & 10                       & 10                                & 1,0                                    & 600                                    \\ \hline
9   & 22                       & 10                                & 2,2                                    & 1320                                   \\ \hline
10  & 22                       & 10                                & 2,2                                    & 1320                                   \\ \hline
\multicolumn{4}{|c|}{Wartość średnia $\overline{S}_V$}                                                      & 1038,0                                   \\ \hline
\multicolumn{4}{|c|}{Odchylenie standardowe $s$}                                                            & 343,0                                    \\ \hline
\multicolumn{4}{|c|}{Przedział ufności $\mu$}                                                               & 1038,0$\pm$672,25                      \\ \hline
\end{tabular}%
}
\end{table}

Błąd względny wyniósł:

$$\gamma=0.20$$

\subsubsection{Pomiar orientacji powierzchni granicznych}

W celu obliczenia ilości przecięć na jednostkę długości odcinka wykorzystaliśmy wzór:

$$P_L=\cfrac{P}{L} \cdot p$$

Gdzie:

$P$ - ilość przeciętych granic międzyziarnowych,

$L$ - długość odcinka w cm,

$p$ - powiększenie.
\newline

Do policzenia rozwinięcia powierzchni wykorzystaliśmy wzór:

$$S_V= 2\cdot P_L$$

% Please add the following required packages to your document preamble:
% \usepackage{graphicx}
% \usepackage[table.xcdraw]{xcolor}
% If you use beamer only pass "xcolor=table" option. i.e. \documentclass[xcolor=table]{beamer}
\begin{table}[H]
    \centering
    \caption{Pomiar orientacji powierzchni.}
    \label{tab:my-table}
    \resizebox{0.7\textwidth}{!}{%
    \begin{tabular}{|c|r|r|r|r|r|r|l}
    \cline{1-7}
    Kąt   & \multicolumn{1}{c|}{$0^\circ$} & \multicolumn{1}{c|}{$30^\circ$} & \multicolumn{1}{c|}{$60^\circ$} & \multicolumn{1}{c|}{$90^\circ$} & \multicolumn{1}{c|}{$120^\circ$} & \multicolumn{1}{c|}{$150^\circ$} &                  \\ \cline{1-7}
    1     & 9                              & 5                                                    & 9                                                    & 7                                                    & 15                                                    & 7                                                     &                                       \\ \cline{1-7}
    2     & 5                              & 8                                                    & 8                                                    & 6                                                    & 8                                                     & 2                                                     &                                       \\ \cline{1-7}
    3     & 6                              & 4                                                    & 9                                                    & 6                                                    & 12                                                    & 5                                                     &                                       \\ \cline{1-7}
    4     & 9                              & 11                                                   & 11                                                   & 7                                                    & 9                                                     & 8                                                     &                                       \\ \cline{1-7}
    5     & 9                              & 4                                                    & 8                                                    & 6                                                    & 7                                                     & 5                                                     &                                       \\ \cline{1-7}
    6     & 9                              & 9                                                    & 6                                                    & 6                                                    & 7                                                     & 1                                                     &                                       \\ \cline{1-7}
    7     & 6                              & 8                                                    & 13                                                   & 7                                                    & 9                                                     & 4                                                     &                                       \\ \cline{1-7}
    8     & 5                              & 5                                                    & 13                                                   & 6                                                    & 7                                                     & 1                                                     &                                       \\ \cline{1-7}
    9     & 9                              & 12                                                   & 5                                                    & 6                                                    & 7                                                     & 9                                                     &                                       \\ \cline{1-7}
    10    & 12                             & 6                                                    & 8                                                    & 6                                                    & 8                                                     & 3                                                     & \multicolumn{1}{r}{}                  \\ \hline
    Średnia  & 7.9                             & 7.2                                                   & 9.0                                                   & 6.3                                                   & 8.9                                                    & 4.5                                                    & \multicolumn{1}{c|}{$\overline{P}_L$ i $\overline{S}_V$} \\ \hline
    $P_L$ & 474                           & 432                                                 & 540                                                  & 378                                                 & 534                                                  & 270                                                   & \multicolumn{1}{r|}{438}             \\ \hline
    $S_V$ & 948                           & 864                                                 & 1080                                                  & 756                                                 & 1068                                                  & 540                                                   & \multicolumn{1}{r|}{876}             \\ \hline
    \end{tabular}%
    }
    \end{table}

Odchylenie standardowe dla pomiaru orientacji powierzchni granicznych 

wynosi:
$$s=205.48$$

Przedział ufności dla pomiaru orientacji powierzchni granicznych wynosi:
$$\mu = 876\pm 402.73$$

Błąd względny dla pomiaru orientacji powierzchni granicznych wynosi:
$$\gamma = 0.15$$
\newline

\begin{figure}[H]
    \centering
    \includegraphics[width=\textwidth]{img/Mapka liczby przecięć.pdf}
\end{figure}


Na podstawie mapki przecięć widzimy, że najmniej przecięć jest przy $150^{\circ}$. Oznacza to, że ziarno może być spłaszczone w tym kierunku. Widzimy również, że ziarna nie są ukierunkowane w jedną stronę, tylko w trzy. Dlatego nie można stwierdzić, że ziarna były ułożone podłużnie.  
Ziarna mogą być kuliste.

\subsection{Wielkość ziaren}

Zmierzono średnicę wszystkich nie przeciętych brzegiem zdjęcia ziaren. Wykonano pomiar w dwóch prostopadłych kierunkach, a następnie wyliczono wartość średnią $d$ dla każdego ziarna. Wyniki zestawiono w tabeli 5.

Znaleźliśmy ziarno z największą średnicą $d_{max}$, którym jest ziarno numer 93 i podzieliliśmy tą wielkość na osiem równych części, dzięki czemu ustaliliśmy wartości przedziałów. Pogrupowano wyniki do odpowiednich przedziałów i wyliczono średnicę przekrojów di, która jest środkową wartością każdego przedziału i jej odwrotność. Wykonano pozostałe obliczenia uwzględniając powiększenie p=1200. Wyniki zestawiono w tabeli ....
Następnie wyliczono parametry przestrzenne ziaren: liczbę ziaren na jednostkę objętości $N_V$, średnią wielkość ziaren $\overline{D}$, średnie odchylenie kwadratowe średnic ziaren $\sigma D$ oraz udział objętościowy $V_V$.

Następnie wyliczono parametry przestrzenne ziaren: liczbę ziaren na jednostkę objętości $N_V$, średnią wielkość ziaren $\overline{D}$, średnie odchylenie kwadratowe oraz udział objętościowy $V_V$.

Wartość $d$ policzyliśmy ze wzoru:

$$d=\cfrac{l+m}{2}$$

gdzie:

$l$ i $k$ - wymiary prostopadłych do siebie średnic ziarna

% Please add the following required packages to your document preamble:
% \usepackage{graphicx}
\begin{table}[H]
\centering
\caption{Zestawienie wymiarów ziaren.}
\label{tab:my-table}
\resizebox{\textwidth}{!}{%
\begin{tabular}{|r|r|r|r|l|r|r|r|r|l|r|r|r|r|lrrrr}
\cline{1-4} \cline{6-9} \cline{11-14} \cline{16-19}
\multicolumn{1}{|l|}{nr. ziarna} & \multicolumn{1}{l|}{$l[mm]$} & \multicolumn{1}{l|}{$m[mm]$} & \multicolumn{1}{l|}{$d[mm]$} &  & 46 & 6,0  & 7,0  & 6,5  &  & 92  & 4,0  & 11,0 & 7,5  & \multicolumn{1}{l|}{} & \multicolumn{1}{r|}{138} & \multicolumn{1}{r|}{5,0}  & \multicolumn{1}{r|}{4,0}  & \multicolumn{1}{r|}{4,5}  \\ \cline{1-4} \cline{6-9} \cline{11-14} \cline{16-19} 
1                                & 12,0                         & 13,0                         & 12,5                         &  & 47 & 4,0  & 4,0  & 4,0  &  & 93  & 16,0 & 17,0 & 16,5 & \multicolumn{1}{l|}{} & \multicolumn{1}{r|}{139} & \multicolumn{1}{r|}{5,0}  & \multicolumn{1}{r|}{3,0}  & \multicolumn{1}{r|}{4,0}  \\ \cline{1-4} \cline{6-9} \cline{11-14} \cline{16-19} 
2                                & 6,0                          & 4,0                          & 5,0                          &  & 48 & 9,0  & 8,0  & 8,5  &  & 94  & 15,0 & 15,0 & 15,0 & \multicolumn{1}{l|}{} & \multicolumn{1}{r|}{140} & \multicolumn{1}{r|}{2,0}  & \multicolumn{1}{r|}{4,0}  & \multicolumn{1}{r|}{3,0}  \\ \cline{1-4} \cline{6-9} \cline{11-14} \cline{16-19} 
3                                & 5,0                          & 3,0                          & 4,0                          &  & 49 & 5,0  & 3,0  & 4,0  &  & 95  & 16,0 & 14,0 & 15,0 & \multicolumn{1}{l|}{} & \multicolumn{1}{r|}{141} & \multicolumn{1}{r|}{3,0}  & \multicolumn{1}{r|}{3,0}  & \multicolumn{1}{r|}{3,0}  \\ \cline{1-4} \cline{6-9} \cline{11-14} \cline{16-19} 
4                                & 1,0                          & 1,0                          & 1,0                          &  & 50 & 15,0 & 7,0  & 11,0 &  & 96  & 8,0  & 5,0  & 6,5  & \multicolumn{1}{l|}{} & \multicolumn{1}{r|}{142} & \multicolumn{1}{r|}{3,0}  & \multicolumn{1}{r|}{3,0}  & \multicolumn{1}{r|}{3,0}  \\ \cline{1-4} \cline{6-9} \cline{11-14} \cline{16-19} 
5                                & 5,0                          & 7,0                          & 6,0                          &  & 51 & 7,0  & 6,0  & 6,5  &  & 97  & 6,0  & 4,0  & 5,0  & \multicolumn{1}{l|}{} & \multicolumn{1}{r|}{143} & \multicolumn{1}{r|}{4,0}  & \multicolumn{1}{r|}{6,0}  & \multicolumn{1}{r|}{5,0}  \\ \cline{1-4} \cline{6-9} \cline{11-14} \cline{16-19} 
6                                & 3,0                          & 5,0                          & 4,0                          &  & 52 & 7,0  & 5,0  & 6,0  &  & 98  & 5,0  & 8,0  & 6,5  & \multicolumn{1}{l|}{} & \multicolumn{1}{r|}{144} & \multicolumn{1}{r|}{6,0}  & \multicolumn{1}{r|}{5,0}  & \multicolumn{1}{r|}{5,5}  \\ \cline{1-4} \cline{6-9} \cline{11-14} \cline{16-19} 
7                                & 2,0                          & 1,0                          & 1,5                          &  & 53 & 9,0  & 6,0  & 7,5  &  & 99  & 5,0  & 4,0  & 4,5  & \multicolumn{1}{l|}{} & \multicolumn{1}{r|}{145} & \multicolumn{1}{r|}{4,0}  & \multicolumn{1}{r|}{7,0}  & \multicolumn{1}{r|}{5,5}  \\ \cline{1-4} \cline{6-9} \cline{11-14} \cline{16-19} 
8                                & 9,0                          & 9,0                          & 9,0                          &  & 54 & 8,0  & 9,0  & 8,5  &  & 100 & 7,0  & 8,0  & 7,5  & \multicolumn{1}{l|}{} & \multicolumn{1}{r|}{146} & \multicolumn{1}{r|}{10,0} & \multicolumn{1}{r|}{5,0}  & \multicolumn{1}{r|}{7,5}  \\ \cline{1-4} \cline{6-9} \cline{11-14} \cline{16-19} 
9                                & 3,0                          & 2,0                          & 2,5                          &  & 55 & 7,0  & 6,0  & 6,5  &  & 101 & 12,0 & 8,0  & 10,0 & \multicolumn{1}{l|}{} & \multicolumn{1}{r|}{147} & \multicolumn{1}{r|}{2,0}  & \multicolumn{1}{r|}{3,0}  & \multicolumn{1}{r|}{2,5}  \\ \cline{1-4} \cline{6-9} \cline{11-14} \cline{16-19} 
10                               & 5,0                          & 4,0                          & 4,5                          &  & 56 & 5,0  & 4,0  & 4,5  &  & 102 & 4,0  & 6,0  & 5,0  & \multicolumn{1}{l|}{} & \multicolumn{1}{r|}{148} & \multicolumn{1}{r|}{7,0}  & \multicolumn{1}{r|}{2,0}  & \multicolumn{1}{r|}{4,5}  \\ \cline{1-4} \cline{6-9} \cline{11-14} \cline{16-19} 
11                               & 4,0                          & 2,0                          & 3,0                          &  & 57 & 5,0  & 3,0  & 4,0  &  & 103 & 16,0 & 13,0 & 14,5 & \multicolumn{1}{l|}{} & \multicolumn{1}{r|}{149} & \multicolumn{1}{r|}{2,0}  & \multicolumn{1}{r|}{2,0}  & \multicolumn{1}{r|}{2,0}  \\ \cline{1-4} \cline{6-9} \cline{11-14} \cline{16-19} 
12                               & 12,0                         & 10,0                         & 11,0                         &  & 58 & 8,0  & 7,0  & 7,5  &  & 104 & 11,0 & 10,0 & 10,5 & \multicolumn{1}{l|}{} & \multicolumn{1}{r|}{150} & \multicolumn{1}{r|}{5,0}  & \multicolumn{1}{r|}{5,0}  & \multicolumn{1}{r|}{5,0}  \\ \cline{1-4} \cline{6-9} \cline{11-14} \cline{16-19} 
13                               & 4,0                          & 2,0                          & 3,0                          &  & 59 & 5,0  & 4,0  & 4,5  &  & 105 & 6,0  & 4,0  & 5,0  & \multicolumn{1}{l|}{} & \multicolumn{1}{r|}{151} & \multicolumn{1}{r|}{6,0}  & \multicolumn{1}{r|}{5,0}  & \multicolumn{1}{r|}{5,5}  \\ \cline{1-4} \cline{6-9} \cline{11-14} \cline{16-19} 
14                               & 5,0                          & 6,0                          & 5,5                          &  & 60 & 6,0  & 7,0  & 6,5  &  & 106 & 6,0  & 6,0  & 6,0  & \multicolumn{1}{l|}{} & \multicolumn{1}{r|}{152} & \multicolumn{1}{r|}{10,0} & \multicolumn{1}{r|}{9,0}  & \multicolumn{1}{r|}{9,5}  \\ \cline{1-4} \cline{6-9} \cline{11-14} \cline{16-19} 
15                               & 7,0                          & 7,0                          & 7,0                          &  & 61 & 7,0  & 5,0  & 6,0  &  & 107 & 5,0  & 6,0  & 5,5  & \multicolumn{1}{l|}{} & \multicolumn{1}{r|}{153} & \multicolumn{1}{r|}{10,0} & \multicolumn{1}{r|}{5,0}  & \multicolumn{1}{r|}{7,5}  \\ \cline{1-4} \cline{6-9} \cline{11-14} \cline{16-19} 
16                               & 3,0                          & 2,0                          & 2,5                          &  & 62 & 7,0  & 5,0  & 6,0  &  & 108 & 7,0  & 6,0  & 6,5  & \multicolumn{1}{l|}{} & \multicolumn{1}{r|}{154} & \multicolumn{1}{r|}{3,0}  & \multicolumn{1}{r|}{4,0}  & \multicolumn{1}{r|}{3,5}  \\ \cline{1-4} \cline{6-9} \cline{11-14} \cline{16-19} 
17                               & 6,0                          & 3,0                          & 4,5                          &  & 63 & 5,0  & 5,0  & 5,0  &  & 109 & 8,0  & 7,0  & 7,5  & \multicolumn{1}{l|}{} & \multicolumn{1}{r|}{155} & \multicolumn{1}{r|}{4,0}  & \multicolumn{1}{r|}{3,0}  & \multicolumn{1}{r|}{3,5}  \\ \cline{1-4} \cline{6-9} \cline{11-14} \cline{16-19} 
18                               & 1,0                          & 1,0                          & 1,0                          &  & 64 & 4,0  & 5,0  & 4,5  &  & 110 & 6,0  & 6,0  & 6,0  & \multicolumn{1}{l|}{} & \multicolumn{1}{r|}{156} & \multicolumn{1}{r|}{6,0}  & \multicolumn{1}{r|}{5,0}  & \multicolumn{1}{r|}{5,5}  \\ \cline{1-4} \cline{6-9} \cline{11-14} \cline{16-19} 
19                               & 7,0                          & 3,0                          & 5,0                          &  & 65 & 4,0  & 4,0  & 4,0  &  & 111 & 4,0  & 8,0  & 6,0  & \multicolumn{1}{l|}{} & \multicolumn{1}{r|}{157} & \multicolumn{1}{r|}{3,0}  & \multicolumn{1}{r|}{2,0}  & \multicolumn{1}{r|}{2,5}  \\ \cline{1-4} \cline{6-9} \cline{11-14} \cline{16-19} 
20                               & 8,0                          & 7,0                          & 7,5                          &  & 66 & 4,0  & 4,0  & 4,0  &  & 112 & 6,0  & 5,0  & 5,5  & \multicolumn{1}{l|}{} & \multicolumn{1}{r|}{158} & \multicolumn{1}{r|}{5,0}  & \multicolumn{1}{r|}{5,0}  & \multicolumn{1}{r|}{5,0}  \\ \cline{1-4} \cline{6-9} \cline{11-14} \cline{16-19} 
21                               & 9,0                          & 6,0                          & 7,5                          &  & 67 & 4,0  & 5,0  & 4,5  &  & 113 & 6,0  & 5,0  & 5,5  & \multicolumn{1}{l|}{} & \multicolumn{1}{r|}{159} & \multicolumn{1}{r|}{3,0}  & \multicolumn{1}{r|}{3,0}  & \multicolumn{1}{r|}{3,0}  \\ \cline{1-4} \cline{6-9} \cline{11-14} \cline{16-19} 
22                               & 6,0                          & 4,0                          & 5,0                          &  & 68 & 5,0  & 8,0  & 6,5  &  & 114 & 7,0  & 6,0  & 6,5  & \multicolumn{1}{l|}{} & \multicolumn{1}{r|}{160} & \multicolumn{1}{r|}{5,0}  & \multicolumn{1}{r|}{3,0}  & \multicolumn{1}{r|}{4,0}  \\ \cline{1-4} \cline{6-9} \cline{11-14} \cline{16-19} 
23                               & 8,0                          & 9,0                          & 8,5                          &  & 69 & 3,0  & 3,0  & 3,0  &  & 115 & 6,0  & 6,0  & 6,0  & \multicolumn{1}{l|}{} & \multicolumn{1}{r|}{161} & \multicolumn{1}{r|}{6,0}  & \multicolumn{1}{r|}{5,0}  & \multicolumn{1}{r|}{5,5}  \\ \cline{1-4} \cline{6-9} \cline{11-14} \cline{16-19} 
24                               & 6,0                          & 7,0                          & 6,5                          &  & 70 & 3,0  & 2,0  & 2,5  &  & 116 & 10,0 & 5,0  & 7,5  & \multicolumn{1}{l|}{} & \multicolumn{1}{r|}{162} & \multicolumn{1}{r|}{8,0}  & \multicolumn{1}{r|}{5,0}  & \multicolumn{1}{r|}{6,5}  \\ \cline{1-4} \cline{6-9} \cline{11-14} \cline{16-19} 
25                               & 6,0                          & 7,0                          & 6,5                          &  & 71 & 3,0  & 5,0  & 4,0  &  & 117 & 5,0  & 5,0  & 5,0  & \multicolumn{1}{l|}{} & \multicolumn{1}{r|}{163} & \multicolumn{1}{r|}{4,0}  & \multicolumn{1}{r|}{2,0}  & \multicolumn{1}{r|}{3,0}  \\ \cline{1-4} \cline{6-9} \cline{11-14} \cline{16-19} 
26                               & 7,0                          & 5,0                          & 6,0                          &  & 72 & 4,0  & 3,0  & 3,5  &  & 118 & 5,0  & 6,0  & 5,5  & \multicolumn{1}{l|}{} & \multicolumn{1}{r|}{164} & \multicolumn{1}{r|}{9,0}  & \multicolumn{1}{r|}{8,0}  & \multicolumn{1}{r|}{8,5}  \\ \cline{1-4} \cline{6-9} \cline{11-14} \cline{16-19} 
27                               & 11,0                         & 8,0                          & 9,5                          &  & 73 & 3,0  & 2,0  & 2,5  &  & 119 & 9,0  & 6,0  & 7,5  & \multicolumn{1}{l|}{} & \multicolumn{1}{r|}{165} & \multicolumn{1}{r|}{6,0}  & \multicolumn{1}{r|}{5,0}  & \multicolumn{1}{r|}{5,5}  \\ \cline{1-4} \cline{6-9} \cline{11-14} \cline{16-19} 
28                               & 4,0                          & 3,0                          & 3,5                          &  & 74 & 5,0  & 6,0  & 5,5  &  & 120 & 9,0  & 6,0  & 7,5  & \multicolumn{1}{l|}{} & \multicolumn{1}{r|}{166} & \multicolumn{1}{r|}{7,0}  & \multicolumn{1}{r|}{6,0}  & \multicolumn{1}{r|}{6,5}  \\ \cline{1-4} \cline{6-9} \cline{11-14} \cline{16-19} 
29                               & 4,0                          & 2,0                          & 3,0                          &  & 75 & 6,0  & 6,0  & 6,0  &  & 121 & 8,0  & 8,0  & 8,0  & \multicolumn{1}{l|}{} & \multicolumn{1}{r|}{167} & \multicolumn{1}{r|}{10,0} & \multicolumn{1}{r|}{13,0} & \multicolumn{1}{r|}{11,5} \\ \cline{1-4} \cline{6-9} \cline{11-14} \cline{16-19} 
30                               & 5,0                          & 4,0                          & 4,5                          &  & 76 & 13,0 & 12,0 & 12,5 &  & 122 & 9,0  & 4,0  & 6,5  & \multicolumn{1}{l|}{} & \multicolumn{1}{r|}{168} & \multicolumn{1}{r|}{10,0} & \multicolumn{1}{r|}{6,0}  & \multicolumn{1}{r|}{8,0}  \\ \cline{1-4} \cline{6-9} \cline{11-14} \cline{16-19} 
31                               & 9,0                          & 8,0                          & 8,5                          &  & 77 & 2,0  & 3,0  & 2,5  &  & 123 & 5,0  & 6,0  & 5,5  & \multicolumn{1}{l|}{} & \multicolumn{1}{r|}{169} & \multicolumn{1}{r|}{4,0}  & \multicolumn{1}{r|}{5,0}  & \multicolumn{1}{r|}{4,5}  \\ \cline{1-4} \cline{6-9} \cline{11-14} \cline{16-19} 
32                               & 4,0                          & 5,0                          & 4,5                          &  & 78 & 6,0  & 7,0  & 6,5  &  & 124 & 3,0  & 2,0  & 2,5  & \multicolumn{1}{l|}{} & \multicolumn{1}{r|}{170} & \multicolumn{1}{r|}{5,0}  & \multicolumn{1}{r|}{6,0}  & \multicolumn{1}{r|}{5,5}  \\ \cline{1-4} \cline{6-9} \cline{11-14} \cline{16-19} 
33                               & 6,0                          & 7,0                          & 6,5                          &  & 79 & 10,0 & 12,0 & 11,0 &  & 125 & 3,0  & 2,0  & 2,5  & \multicolumn{1}{l|}{} & \multicolumn{1}{r|}{171} & \multicolumn{1}{r|}{5,0}  & \multicolumn{1}{r|}{7,0}  & \multicolumn{1}{r|}{6,0}  \\ \cline{1-4} \cline{6-9} \cline{11-14} \cline{16-19} 
34                               & 7,0                          & 7,0                          & 7,0                          &  & 80 & 7,0  & 7,0  & 7,0  &  & 126 & 15,0 & 14,0 & 14,5 & \multicolumn{1}{l|}{} & \multicolumn{1}{r|}{172} & \multicolumn{1}{r|}{2,0}  & \multicolumn{1}{r|}{2,0}  & \multicolumn{1}{r|}{2,0}  \\ \cline{1-4} \cline{6-9} \cline{11-14} \cline{16-19} 
35                               & 5,0                          & 9,0                          & 7,0                          &  & 81 & 6,0  & 3,0  & 4,5  &  & 127 & 8,0  & 6,0  & 7,0  & \multicolumn{1}{l|}{} & \multicolumn{1}{r|}{173} & \multicolumn{1}{r|}{6,0}  & \multicolumn{1}{r|}{2,0}  & \multicolumn{1}{r|}{4,0}  \\ \cline{1-4} \cline{6-9} \cline{11-14} \cline{16-19} 
36                               & 13,0                         & 11,0                         & 12,0                         &  & 82 & 12,0 & 3,0  & 7,5  &  & 128 & 5,0  & 3,0  & 4,0  & \multicolumn{1}{l|}{} & \multicolumn{1}{r|}{174} & \multicolumn{1}{r|}{5,0}  & \multicolumn{1}{r|}{8,0}  & \multicolumn{1}{r|}{6,5}  \\ \cline{1-4} \cline{6-9} \cline{11-14} \cline{16-19} 
37                               & 4,0                          & 3,0                          & 3,5                          &  & 83 & 12,0 & 10,0 & 11,0 &  & 129 & 10,0 & 4,0  & 7,0  & \multicolumn{1}{l|}{} & \multicolumn{1}{r|}{175} & \multicolumn{1}{r|}{4,0}  & \multicolumn{1}{r|}{5,0}  & \multicolumn{1}{r|}{4,5}  \\ \cline{1-4} \cline{6-9} \cline{11-14} \cline{16-19} 
38                               & 3,0                          & 4,0                          & 3,5                          &  & 84 & 7,0  & 6,0  & 6,5  &  & 130 & 6,0  & 7,0  & 6,5  & \multicolumn{1}{l|}{} & \multicolumn{1}{r|}{176} & \multicolumn{1}{r|}{10,0} & \multicolumn{1}{r|}{8,0}  & \multicolumn{1}{r|}{9,0}  \\ \cline{1-4} \cline{6-9} \cline{11-14} \cline{16-19} 
39                               & 3,0                          & 1,0                          & 2,0                          &  & 85 & 6,0  & 4,0  & 5,0  &  & 131 & 7,0  & 7,0  & 7,0  & \multicolumn{1}{l|}{} & \multicolumn{1}{r|}{177} & \multicolumn{1}{r|}{6,0}  & \multicolumn{1}{r|}{9,0}  & \multicolumn{1}{r|}{7,5}  \\ \cline{1-4} \cline{6-9} \cline{11-14} \cline{16-19} 
40                               & 6,0                          & 4,0                          & 5,0                          &  & 86 & 5,0  & 3,0  & 4,0  &  & 132 & 7,0  & 4,0  & 5,5  & \multicolumn{1}{l|}{} & \multicolumn{1}{r|}{178} & \multicolumn{1}{r|}{8,0}  & \multicolumn{1}{r|}{3,0}  & \multicolumn{1}{r|}{5,5}  \\ \cline{1-4} \cline{6-9} \cline{11-14} \cline{16-19} 
41                               & 8,0                          & 6,0                          & 7,0                          &  & 87 & 9,0  & 4,0  & 6,5  &  & 133 & 3,0  & 3,0  & 3,0  & \multicolumn{1}{l|}{} & \multicolumn{1}{r|}{179} & \multicolumn{1}{r|}{2,0}  & \multicolumn{1}{r|}{2,0}  & \multicolumn{1}{r|}{2,0}  \\ \cline{1-4} \cline{6-9} \cline{11-14} \cline{16-19} 
42                               & 3,0                          & 4,0                          & 3,5                          &  & 88 & 4,0  & 2,0  & 3,0  &  & 134 & 5,0  & 6,0  & 5,5  & \multicolumn{1}{l|}{} & \multicolumn{1}{r|}{180} & \multicolumn{1}{r|}{7,0}  & \multicolumn{1}{r|}{2,0}  & \multicolumn{1}{r|}{4,5}  \\ \cline{1-4} \cline{6-9} \cline{11-14} \cline{16-19} 
43                               & 6,0                          & 6,0                          & 6,0                          &  & 89 & 9,0  & 7,0  & 8,0  &  & 135 & 7,0  & 7,0  & 7,0  & \multicolumn{1}{l|}{} & \multicolumn{1}{r|}{181} & \multicolumn{1}{r|}{7,0}  & \multicolumn{1}{r|}{5,0}  & \multicolumn{1}{r|}{6,0}  \\ \cline{1-4} \cline{6-9} \cline{11-14} \cline{16-19} 
44                               & 6,0                          & 5,0                          & 5,5                          &  & 90 & 5,0  & 9,0  & 7,0  &  & 136 & 8,0  & 3,0  & 5,5  & \multicolumn{1}{l|}{} & \multicolumn{1}{r|}{182} & \multicolumn{1}{r|}{5,0}  & \multicolumn{1}{r|}{6,0}  & \multicolumn{1}{r|}{5,5}  \\ \cline{1-4} \cline{6-9} \cline{11-14} \cline{16-19} 
45                               & 5,0                          & 5,0                          & 5,0                          &  & 91 & 7,0  & 4,0  & 5,5  &  & 137 & 10,0 & 13,0 & 11,5 &                       & \multicolumn{1}{l}{}     & \multicolumn{1}{l}{}      & \multicolumn{1}{l}{}      & \multicolumn{1}{l}{}      \\ \cline{1-4} \cline{6-9} \cline{11-14}
\end{tabular}%
}
\end{table}

% Please add the following required packages to your document preamble:
% \usepackage{graphicx}
\begin{table}[H]
\centering
\caption{Zestawienie wyników pomiarów i~obliczeń.}
\label{tab:my-table}
\resizebox{\textwidth}{!}{%
\begin{tabular}{|c|c|r|r|r|r|r|r|}
\hline
\begin{tabular}[c]{@{}c@{}}Numer \\ klasy\\ przedziału\end{tabular} & Przedział $[\mu m]$         & \multicolumn{1}{c|}{$d_i[\mu m]$} & \multicolumn{1}{c|}{$\cfrac{1}{d_i}[\mu m]$} & \multicolumn{1}{c|}{$N_i$} & \multicolumn{1}{c|}{$N_i\cdot d_i [\mu m]$} & \multicolumn{1}{c|}{$\cfrac{n_i}{d_i} [\mu m^{-1}]$} & \multicolumn{1}{c|}{$\cfrac{p_i\cdot N_i\cdot d_i^2}{4} [\mu m^2]$} \\ \hline
1                                                                   & (0-2,06\textgreater{}      & 1,03                           & 0,97                                      & 7                          & 7,21                                     & 6,80                                              & 5,83                                                             \\ \hline
2                                                                   & (2,06-4,12\textgreater{}   & 3,09                           & 0,32                                      & 40                         & 123,60                                   & 12,94                                             & 299,81                                                           \\ \hline
3                                                                   & (4,12-6,18\textgreater{}   & 5,15                           & 0,19                                      & 63                         & 324,45                                   & 12,23                                             & 1311,67                                                          \\ \hline
4                                                                   & (6,18-8,24\textgreater{}   & 7,21                           & 0,14                                      & 47                         & 338,87                                   & 6,52                                              & 1917,95                                                          \\ \hline
5                                                                   & (8,24-10,3\textgreater{}   & 9,27                           & 0,11                                      & 10                         & 92,70                                    & 1,08                                              & 674,57                                                           \\ \hline
6                                                                   & (10,30-12,36\textgreater{} & 11,33                          & 0,09                                      & 8                          & 90,64                                    & 0,71                                              & 806,16                                                           \\ \hline
7                                                                   & (12,36-14,42\textgreater{} & 13,39                          & 0,07                                      & 3                          & 40,17                                    & 0,22                                              & 422,23                                                           \\ \hline
8                                                                   & (14,92-16,5\textgreater{}  & 15,95                          & 0,06                                      & 4                          & 63,80                                    & 0,25                                              & 798,82                                                           \\ \hline
\multicolumn{4}{|c|}{suma}                                                                                                                                                    & 182                        & 1081,44                                  & 40,75                                             & 6237,05                                                          \\ \hline
\end{tabular}%
}
\end{table}

Pozostałych obliczeń dokonaliśmy za pomocą wzorów:

$$\overline{d}=\cfrac{\sum N_id_i}{N}$$

$$\overline{d}=5.94[\mu m]$$

$$\overline{d^{-1}}=\cfrac{\sum N_id_i^{-1}}{N}$$

$$\overline{d^{-1}}=0.22[\mu m^{-1}]$$

$$N_A=\cfrac{p^2}{a\cdot b}(z+0.5w+0.25u)$$

$$N_A=\cfrac{1200^2}{102\cdot 80}(216+0.5\cdot 65+0.25\cdot 2)=88374672.16[mm^{-2}]$$

Gdzie:

$\overline{d}$ - średnią średnicę

$\overline{d^{-1}}$ - odwrotność średniej średnicy

$N$ - suma liczebności

$N_A$ - liczba ziaren mierzonych na jednostkowej powierzchni zgładu,

$p$ - powiększenie mikroskopu,

$a$ i $b$ - długość boków prostokąta,

$z$ - liczba ziaren leżących całkowicie wewnątrz prostokąta,

$w$ - liczba ziaren przeciętych przez brzegi prostokąta,

$u$ - liczba ziaren leżących w narożach prostokąta.

$$N_V=\cfrac{2}{\pi}\cdot \overline{d^{-1}} \cdot N_A$$

$$N_V=88341.08[mm^{-3}]$$

$$\overline{D}=\cfrac{\pi}{2}\cdot \overline{d^{-1}}^{-1}$$

$$\overline{D}=7.01[\mu m]$$

$$\sigma_D=\sqrt{\cfrac{4}{\pi}\cdot \overline{d}\cdot \overline{D} - \overline{D}^2}$$

$$\sigma_D=1.98 [\mu m]$$

$$A=\cfrac{a\cdot b}{p^2}$$

$$A=0.005666[mm^2]=5666[\mu m^2]$$

$$V_V=\cfrac{\sum N_id_i^2}{4\cdot A}$$

$$V_V=51.60$$

Gdzie:

$N_V$ - liczba ziaren na jednostkę objętości

$\overline{D}$ - średnia wielkość ziaren

$\sigma_D$ - średnie odchylenie kwadratowe średnic ziaren

$A$ - pole powierzchni fotografii

$V_V$ - udział objętościowy

\section{Podsumowanie i wnioski}

\subsection{Udział objętościowy $V_V$}

% Please add the following required packages to your document preamble:
% \usepackage{graphicx}
\begin{table}[H]
\centering
\caption{Porównanie wyników udziału objętościowego $V_V$.}
\label{tab:my-table}
\resizebox{0.7\textwidth}{!}{%
\begin{tabular}{|c|r|r|r|r|}
\hline
Metoda   & \multicolumn{1}{c|}{$V_V(\alpha)$} & \multicolumn{1}{c|}{$\mu$} & \multicolumn{1}{c|}{$n$} & \multicolumn{1}{c|}{$\gamma$} \\ \hline
Punktowa & 0.16                                               & 0.16$\pm$0.0541            & 193                      & 0.1073                        \\ \hline
Liniowa  & 0.23                                               & 0.23$\pm$0.1211            & 51                       & 0.1651                        \\ \hline
\end{tabular}%
}
\end{table}

Obydwie metody, zarówno metoda liniowa jak i punktowa, dały podobne  do siebie wyniki wartości udziału fazy $\alpha$ (różnią się wartością około $0.07$), co może świadczyć, że metody są do siebie zbliżone. W przypadku metody punktowej błąd względny wynosi  około $0.10$, nie jest to duży błąd, świadczy to  o dokładności przeprowadzonych obliczeń, natomiast w metodzie liniowej błąd względny wynosi  $0.16$, jest on wyższy niż w metodzie punktowej. Przyczyną mogła być niedokładność  podczas wykonywania  ćwiczenia spowodowana niepoprawnym odczytaniem wyników lub złego przyłożenia linijki.

\subsection{Rozwinięcie powierzchni $S_V$}

% Please add the following required packages to your document preamble:
% \usepackage{graphicx}
\begin{table}[H]
\centering
\caption{Porównanie wyników rozwinięcia powierzchni $S_V$.}
\label{tab:my-table}
\resizebox{\textwidth}{!}{%
\begin{tabular}{|c|r|r|r|r|}
\hline
Metoda                  & \multicolumn{1}{c|}{$\overline{S}_V$} & \multicolumn{1}{c|}{$s$} & \multicolumn{1}{c|}{$\mu$} & \multicolumn{1}{c|}{$\gamma$} \\ \hline
Siecznych przypadkowych & 103.8                                 & 34.30                    & $103.8\pm 67.23$           & 0.20                          \\ \hline
Siecznych skierowanych  & 876                                   & 205.48                   & $876\pm 402.73$            & 0.15                          \\ \hline
\end{tabular}%
}
\end{table}

Dzięki znajomości ilości przecięć na dany odcinek oraz uwzględniając powiększenie otrzymanej mikrofotografii, mogłyśmy obliczyć rozwinięcie powierzchni materiału porowatego. Niektóre z obliczonych wartości znacznie się od siebie różnią, te rozbieżności spowodowały tak duże wartości przedziału ufności i odchylenia standardowego. W metodzie siecznych przypadkowych rozwiniecie powierzchni jest większe niż w przypadku metodzie siecznych skierowanych.

\subsection{Wielkość ziaren}

\section{Załączniki wyników}

\end{document}
